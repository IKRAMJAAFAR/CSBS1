\documentclass[pt]{beamer}
\hypersetup{pdfpagemode=FullScreen}
\usetheme{CambridgeUS}
\usecolortheme{dolphin}
\usefonttheme{serif}
\usepackage{microtype}
\usepackage[utf8]{inputenc}
\usepackage{graphicx}
\usepackage{mathtools}
\usepackage{breqn}
\usepackage{empheq}
\usepackage{tensor}
\usepackage{array}
\usepackage{multirow}
\usepackage{transparent}
\usepackage{fontenc}
\usepackage{booktabs}
\usepackage{natbib}
\usepackage{hyperref}

% set colors
\definecolor{myNewColorA}{RGB}{39,52,139}
\definecolor{myNewColorB}{RGB}{226,182,0}
\definecolor{myNewColorC}{RGB}{226,182,0}
\setbeamercolor*{palette primary}{bg=myNewColorA, fg=white}
\setbeamercolor*{palette secondary}{bg=myNewColorB, fg=black}
\setbeamercolor*{palette tertiary}{bg=myNewColorB, fg=black}
\setbeamercolor*{titlelike}{fg=myNewColorA}
\setbeamercolor*{title}{bg=myNewColorA, fg=white}
\setbeamercolor*{item}{fg=myNewColorA}
\setbeamercolor*{caption name}{fg=myNewColorA}

% Ensures consistent numbering style
\setbeamertemplate{enumerate items}[circle]

% Adjust indentation for enumerate
\setlength{\leftmargini}{1.5em} 

% define environment
\newenvironment{tres important}[2][]{
	\setkeys{EmphEqEnv}{#2}
	\setkeys{EmphEqOpt}{box={\setlength{\fboxsep}{10pt}\fcolorbox{myNewColorA}{white}},#1}
	\EmphEqMainEnv}
{\endEmphEqMainEnv}

%------------------------------------------------------------
\titlegraphic{
    \vspace{-7.8cm} % Move the logo closer to the top of the slide
    \includegraphics[height=2cm]{img-logo-UM} % Adjust size if needed
    \vspace{0.5cm} % Space between logo and title
}

\setbeamerfont{title}{size=\large}
\setbeamerfont{subtitle}{size=\small}
\setbeamerfont{date}{size=\small}
\setbeamerfont{institute}{size=\large}
\title[Utilizing Numerical Integration in Deep Learning for Forecasting in Time-Series Datasets]{Utilizing Numerical Integration in Deep Learning for Forecasting in Time-Series Datasets}
\subtitle{Group CSBS1}
\institute[WID3015 Numerical Analysis]{Supervisor: Dr Suzan J. Obaiys}
\date[Universiti Malaya]

% This block of commands puts the table of contents at the beginning of each section and highlights the current section:
\AtBeginSection[]
{
  \begin{frame}
    \frametitle{Table of Contents}
    \tableofcontents[currentsection]
  \end{frame}
}
\AtBeginSection[]
{
  \begin{frame}
  \vfill
  \centering
  \begin{beamercolorbox}[sep=8pt,center,shadow=true,rounded=true]{title}
    \usebeamerfont{title}\insertsectionhead\par%
  \end{beamercolorbox}
  \vfill
  \end{frame}
}

%------------------------------------------------------------
\begin{document}

% The title page
\begin{frame}
    \titlepage
\end{frame}

% Table of contents
\begin{frame}
    \frametitle{Table of Contents}
    \tableofcontents
\end{frame}

% Group Member List Frame
\section{Group Member List}
\begin{frame}{Group Member List}
    \begin{table}
    \begin{center}
    \small
        \begin{tabular}{|l|c|}
            \hline
            \textbf{Name} & \textbf{Matrix Number} \\ \hline
            BELDON LIM KAI YI  & 22059390  \\ \hline
            CHONG JIA YING  & U2102853  \\ \hline
            HUMYRA TASMIA  & S2176677  \\ \hline
            MASYITAH HUMAIRA BINTI MOHD HAFIDZ  & U2000518  \\ \hline
            MUHAMMAD BAKHTIAR BIN MOHAMAD  & U2100679 \\
            HARUN KAMAL  &  \\ \hline
            MUHAMMAD IKRAM BIN JAAFAR  & U2100632  \\ \hline
        \end{tabular}
    \end{center}
    \end{table}
\end{frame}

% Introduction Section
\section{Introduction}
\begin{frame}{Introduction}
\begin{itemize}
    \item The usual fields applying the tasks:
    \begin{itemize}
        \item Marketing
        \item Weather
        \item Traffic
    \end{itemize}
    \item The challenges in forecasting
    \item Similar models and their limitations:
    \begin{itemize}
        \item ARIMA and SARIMA - Sensitive to anomalies
        \item Statistical Methods (z-scores, rolling statistics) - Limited to linear patterns
    \end{itemize}
    \item A brief overview of how numerical integration contributes to forecasting.
\end{itemize}
\end{frame}


% Problem Statement and Hypothesis Section
\section{Problem Statement and Hypothesis}
\begin{frame}{Problem Statement and Hypothesis}
\begin{itemize}
    \item Problem Statement \begin{itemize}
        \item The problem addressed in this study is the accuracy and efficiency of forecasting time series data using hybrid numerical integration and deep learning models.
\end{itemize}
\end{frame}

% Objectives Section
\section{Objective}
\begin{frame}{Objective}
    \item Objective \begin{itemize}
    \vspace{1em}
        \item To leverage a predefined time-series dataset to address gaps in forecasting by specified techniques.
        \item To develop and evaluate the deep learning model based on LSTM and numerical integration.
        \item To compare the models using the MAE, MSE, MSLE, $R^2$, IA, MAPE and SMAPE metrics.
    \end{itemize}
\end{frame}

% Methodology Section
\section{Methodology}

% First Page of Methodology
\begin{frame}{Methodology}
\begin{itemize}
    \item[1.] \textbf{Dataset collection}
    \begin{itemize}
        \item NYC taxi passengers
        \item Why we choose this and why is it suitable:
        \begin{itemize}
            \item Type
            \item Properties
            \item Relation to other course assignment
            \item Anomaly
        \end{itemize}
    \end{itemize}

    \item[2.] \textbf{Understanding the numerical Integration methods}
    \begin{itemize}
        \item Define selected methods:
        \begin{itemize}
            \item Trapezoidal Rule
            \item Monte Carlo
        \end{itemize}
        \item Explain how numerical integration results will be used to forecast or preprocess data for deep learning models:
        \begin{itemize}
            \item Integration outputs:
            \begin{itemize}
                \item Cumulative sums
                \item Rate of change
            \end{itemize}
            \item Numerical integration is applied to calculate cumulative trends over time, which serve as additional features for the LSTM model.
        \end{itemize}
    \end{itemize}
\end{itemize}
\end{frame}

% Second Page of Methodology
\begin{frame}{Methodology (cont'd)}
\begin{itemize}
    \item[3.] \textbf{Data preprocessing \& preparation}
    \begin{itemize}
        \item Time-series preprocessing:
        \begin{itemize}
            \item Trend extraction
            \item Seasonality Analysis
        \end{itemize}
        \item Describe the steps in more detail:
        \begin{itemize}
            \item Handling missing values.
            \item Normalizing/standardizing data.
            \item Splitting into training and testing datasets.
        \end{itemize}
    \end{itemize}

    \item[4.] \textbf{Understanding deep learning models structures}
    \begin{itemize}
        \item LSTM:
        \begin{itemize}
            \item Explain its capability.
        \end{itemize}
    \end{itemize}
\end{itemize}
\end{frame}

% Third Page of Methodology
\begin{frame}{Methodology (cont'd)}
\begin{itemize}
    \item[5.] \textbf{Process of Hybrid Models}
    \begin{itemize}
        \item Specify how you’ll combine numerical integration with deep learning:
        \begin{itemize}
            \item Use integration results as features for the deep learning models.
            \item Apply numerical integration to validate predictions or enhance training.
        \end{itemize}
        \item Clarify the process of combining outputs from numerical integration with deep learning:
        \begin{itemize}
            \item Are NI results used directly as input features?
            \item Are they applied to preprocess or smooth the data?
        \end{itemize}
    \end{itemize}

    \item[6.] \textbf{Performance Metrics / Error Analysis}
    \begin{itemize}
        \item Include error analysis methods:
        \begin{itemize}
            \item MAE
            \item MSE
            \item MSLE
            \item $R^2$
            \item IA
            \item MAPE and SMAPE
        \end{itemize}
    \end{itemize}

    \item[7.] \textbf{Discuss about the whole procedure of the project}
    \begin{itemize}
        \item Have not been discussed yet.
    \end{itemize}
\end{itemize}
\end{frame}

% Results and Discussion Section
\section{Results and Discussion}

% First Page of Results and Discussion
\begin{frame}{Results and Discussion}
\begin{itemize}
    \item \textbf{Comparative Analysis}
    \begin{itemize}
        \item Present a clear comparison of Numerical Integration (NI), Deep Learning (DL), and the Hybrid model.
        \item Use:
        \begin{itemize}
            \item Bar charts for metric comparison.
            \item Line plots for actual and predicted.
            \item Confidence intervals.
        \end{itemize}
        \item Explain Results
            \begin{itemize}
            \item Discuss why the hybrid model performs better or worse in specific scenarios.
            \item Relate findings to challenges mentioned in the Introduction.
            \end{itemize}
        \item Visualizations
            \begin{itemize}
            \item Incorporate MATLAB visualizations as outlined earlier to enhance clarity.
            \item Relate findings to challenges mentioned in the Introduction.
            \end{itemize}
    \end{itemize}
\end{itemize}
\end{frame}

% Second Page of Results and Discussion
\begin{frame}{Results and Discussion (cont'd)}
    \begin{itemize}
    \item \textbf{Ensure clear comparisons:}
        \begin{itemize}
            \item Standalone numerical integration vs. standalone deep learning vs. hybrid.
            \item Use line plots to highlight the difference between actual and predicted values.
            \item Include confidence intervals to visualize prediction uncertainty.
        \end{itemize}
    \end{itemize}
\end{frame}

% Conclusion Section
\section{Conclusion }
\begin{frame}{Conclusion}
    \begin{itemize}
    \item \textbf{Ensure the conclusion explicitly addresses the hypothesis:}
        \begin{itemize}
            \item Did the hybrid approach improve forecasting accuracy?
            \item Were there trade-offs, such as increased computational cost?
        \end{itemize}
    \end{itemize}
\end{frame}

% Future Work Section
\section{Future Work}
\begin{frame}{Future Work}
    \begin{itemize}
        \item Suggest exploring real-time forecasting applications or scaling the hybrid approach to larger datasets.
        \item Investigate Anomaly Detection using these numerical methods.
    \end{itemize}
\end{frame}

% Bibliography Section
\section{Bibliography}
\begin{frame}{Bibliography}
    \bibliographystyle{apalike}
    \bibliography{references}
\end{frame}

% References Page
\begin{frame}{References}
    \begin{itemize}
        \item Reference 1:
        \cite{10583885}
        \item Reference 2: 
        \cite{WAQAS2024}
        \item Reference 3: 
        \cite{unknown}
        \item Reference 4: 
        \cite{article}
        \item Reference 5:
        \cite{numerical_analysis}
        \item Reference 6:
        \cite{harvard_unit27}
    \end{itemize}
\end{frame}

% Appendix Section
\section{Appendix}
\begin{frame}{Appendix}
    Appendix content.
\end{frame}

\end{document}
